%----------------------------------------------------------------------------------------
%	PACKAGES AND OTHER DOCUMENT CONFIGURATIONS
%----------------------------------------------------------------------------------------
\documentclass{resume}

\usepackage[left=0.75in,top=0.6in,right=0.75in,bottom=0.6in]{geometry}
\usepackage[T1]{fontenc}
\usepackage[english]{babel}
\usepackage{microtype}
\usepackage{lmodern}
\usepackage{enumitem}
\usepackage{multicol}
\usepackage{booktabs}

\newcommand{\tab}[1]{\hspace{.2667\textwidth}\rlap{#1}}
\newcommand{\itab}[1]{\hspace{0em}\rlap{#1}}

\name{Gustavo Leite}
\address{Av. Albert Einstein, 1251, Cidade Universitária, Campinas-SP, Brazil, ZIP 13083-852}
\address{+55 (19) 99721-4443 \\ {\tt contact@gustavoleite.me}}

\begin{document}

%----------------------------------------------------------------------------------------
%	EDUCATION SECTION
%----------------------------------------------------------------------------------------

\begin{rSection}{Education}
{\bf BSc in Computer Science} \hfill {\em 2013--2016} \\
{\sc São Paulo State University (UNESP)} \hfill {\em Rio Claro-SP, Brazil} \\
I finished the course with grade 8.58 out of 10, placed as the first student of
the class. I developed a study on parallel computing with application to the
$n$-body problem as my final paper. This work was supervised by Prof. Alexandro
Baldassin.

{\bf MSc in Computer Science} \hfill {\em 2017--2019} \\
{\sc São Paulo State University (UNESP)} \hfill {\em Rio Claro-SP, Brazil} \\
I obtained grade A in 5 out of 6 courses attended. I defended the thesis titled
``Performance Evaluation of Code Optimizations in FPGA Accelerators'' in August,
2019 under supervision of Prof. Alexandro Baldassin.

{\bf Complementary Training} \hfill {\em 2018} \\
{\sc University of Alberta} \hfill {\em Edmonton-AB, Canada} \\
During a research internship, I audited two courses on compilers in the
University of Alberta. The courses were ``Compiler Design'' and ``Machine
Learning in Optimizing Compilers'', both lectured by Prof. José Nelson Amaral
during the Fall of 2018.

{\bf PhD in Computer Science} \hfill {\em 2020--present} \\
{\sc University of Campinas (Unicamp)} \hfill {\em Campinas-SP, Brazil} \\
I am currently a PhD candidate at the University of Campinas, working on the
Computer Systems Laboratory (LSC) under the supervision of Prof. Guido Araújo
and Prof. Marcio Machado Pereira.
\end{rSection}

%----------------------------------------------------------------------------------------
%	SKILLS SECTION
%----------------------------------------------------------------------------------------

\begin{rSection}{Awards and Distinctions}
{\bf Best Academic Performance} $\diamond$ {UNESP} \hfill{\em 2017} \\
In March 2017, I was awarded a distinction from the São Paulo State University
for the best academic performance in the Bachelors in Computer Science 2016's
class.

{\bf Winner of the Parallel Programming Challenge} $\diamond$ {ERAD-SP, ICMC-USP} \hfill{\em 2017} \\
In April 2017, my team won the Parallel Programming Challenge carried during the
8th Regional School of High-Performance (ERAD-SP), ICMC-USP, São Carlos-SP,
Brazil.
\end{rSection}

\begin{rSection}{Interests}
Following, a non-exhaustive list of my professional and research interests:

\setlength\multicolsep{5pt}
\begin{multicols}{2}
\begin{itemize}[noitemsep]
  \item high-performance computing;
  \item compilers;
  \item machine learning systems;
  \item computer architecture.
\end{itemize}
\end{multicols}

\end{rSection}

%----------------------------------------------------------------------------------------
%	EXPERIENCE SECTION
%----------------------------------------------------------------------------------------

\begin{rSection}{Professional Experience}
{\it I do not have professional experience.}
\end{rSection}

%--------------------------------------------------------------------------------
%    PROJECTS
%-----------------------------------------------------------------------------------------------

\begin{rSection}{Research Experience}
{\bf Self-Healing Software} \hfill {\em 2014--2015} \\
{\sc São Paulo State University (UNESP)} \hfill {\em Rio Claro-SP, Brazil} \\
During the bachelors course, I have participated in two research projects aimed
at developing better recommendation systems to assist decision-making in
self-healing systems. I was supervised by Prof. Frank José Affonso and received
funding from CNPq and FAPESP. During this period I coauthored two papers
published in international conferences~\cite{SEKE16,SEKE15}.

{\bf Parallel Programming} \hfill {\em 2016} \\
{\sc São Paulo State University (UNESP)} \hfill {\em Rio Claro-SP, Brazil} \\
I have developed a term paper where I studied the parallel programming paradigm
using OpenMP and OpenCL. With the knowledge obtained, I compared the performance
of implementations in CPU and GPU of the $n$-body problem. This work
was supervised by Prof. Alexandro Baldassin.

{\bf High-Performance Computing} \hfill {\em 2017--2019} \\
{\sc São Paulo State University (UNESP)} \hfill {\em Rio Claro-SP, Brazil} \\
During the masters course, I have investigated workload balancing techniques for
NUMA systems and, after that, I have analyzed the performance of code
optimizations aimed at FPGA accelerators present in the literature. I defended
the thesis titled ``Performance Evaluation of Compiler Optimizations in FPGA
Accelerators''~\cite{Master19} in August 2019 under the supervision of Prof.
Alexandro Baldassin. This work was carried with funding from CAPES and FAPESP.
During this period I have also been a teacher assistant in the undergrad course
``Microprocessors II''.

{\bf High-Performance Computing} \hfill {\em 2018} \\
{\sc University of Alberta} \hfill {\em Edmonton-AB, Canada} \\
I visited the University of Alberta between September 2018 and November 2018 for
a research internship.  The objetive of this project was to conduct a
bibliographic survey on existing compiler optimizations for FPGA accelerators. I
was co-supervised by Profs. José Nelson Amaral (UAlberta) and Guido Araújo
(IC-Unicamp). This internship was funded by the BEPE/FAPESP program. From this
collaboration, we published a paper in a brazilian conference~\cite{WSCAD19}.

{\bf OmpCluster Programming Model} \hfill {\em 2020--2021} \\
{\sc University of Campinas (Unicamp)} \hfill {\em Campinas-SP, Brazil} \\
During the first two years of my PhD, I was part of a research project that
consisted in extending the OpenMP programming model for working in cluster
environments. I was responsible for implementing the first version of the task
scheduler in the OmpCluster runtime~\cite{ICPP23, SBAC22}. I also created a
benchmarking tool called OmpcBench that standardized the process of collecting
benchmark results for the runtime. This tool was adopted by the entire team and
is continued to be used to this day. During this period I also helped Prof.
Guido by preparing and lecturing classes about CUDA programming in the
``Parallel Programming'' course offered at the Institute of Computing at
Unicamp.

{\bf Machine Learning Systems} \hfill {\em 2022--present} \\
{\sc University of Campinas (Unicamp)} \hfill {\em Campinas-SP, Brazil} \\
At present I have been involved in a research project that aims to find more
efficient techniques for training large deep learning models. More specifically,
we are investigating how to train models in systems with limited memory by
checkpointing and restoring intermediate values. We have devised a dynamic
programming algorithm for partitioning a computational graph into multiple
stages. We optimize the partitioning scheme for minimizing the amount of data
flowing from one stage to the next, thus lowering the footprint required to save
and restore. Additionally, we compress and decompress the data using a lossless
algorithm for increased efficiency. We plan to submit a research paper to a
relevant conference in the following months.

\end{rSection}

%----------------------------------------------------------------------------------------
%	TEACHING SECTION
%----------------------------------------------------------------------------------------

\begin{rSection}{Teaching Experience}

{\bf Introduction Parallel Programming (MO644)} \hfill {\em 2021--2022} \\
{\sc University of Campinas (Unicamp)} \hfill {\em Campinas-SP, Brazil} \\
During the second semester of 2021 I collaborated with Prof. Guido Araújo by
preparing and lecturing classes on parallel programming. More specifically, I
was responsible for the lectures related to the CUDA programming model. The
content included the CUDA programming language, NVIDIA GPU architecture and
memory hierarchy. This course was offered again in the first semester of 2022.

{\bf Machine Learning Under the Hood (MO436)} \hfill {\em 2022} \\
{\sc University of Campinas (Unicamp)} \hfill {\em Campinas-SP, Brazil} \\
During the second semester of 2022 I collaborated with Prof. Guido Araújo by
preparing and lecturing classes on the Google JAX framework for machine learning
and numeric computing. The lectures covered how to use the framework and, more
importantly, how it works under the hood. We have explained how automatic
differentiation was implemented in JAX and how to extend the library with new
operations.

\end{rSection}

%----------------------------------------------------------------------------------------
%	SKILLS SECTION
%----------------------------------------------------------------------------------------

\begin{rSection}{Skills and Language}
{\bf Programming Languages}

\begin{tabular}{ll} Proficient & C/C++, Python, CUDA \\
  Familiar   & Rust \\
\end{tabular}

{\bf Tools} {\small (Ordered by familiarity)}
\begin{itemize}[noitemsep]
  \item Linux operating system and utilities ({\tt grep}, {\tt bash}, {\tt awk},
    {\tt make}, etc);
  \item C/C++ compilers ({\tt gcc} and {\tt clang});
  \item Control version system ({\tt git}) and platforms (Github, Gitlab, etc);
  \item Parallel programming paradigms (OpenMP, CUDA);
  \item Python packages ({\tt numpy}, {\tt matplotlib}, {\tt seaborn}, etc);
  \item Debuggers ({\tt gdb}) and profilers ({\tt perf}, {\tt strace}, {\tt
    ltrace});
  \item LLVM compiler infrastructure.
\end{itemize}

{\bf Languages Proficiency}
\begin{itemize}[noitemsep]
  \item {\bf Portuguese} (Native) -- comprehension: good; speaking: good; writing: good;
  \item {\bf English} (Fluent) -- comprehension: good; speaking: good; writing: good;
  \item {\bf German} (Beginner/A1) -- comprehension: basic; speaking: none; writing: basic;
\end{itemize}
\end{rSection}

\begin{rSection}{Event Attendance}
{\bf SBAC-PAD'17} $\diamond$ Campinas-SP, Brazil \hfill {\em 2017} \\
International Symposium on Computer Architecture and High Performance
Computing

{\bf 8ª ERAD-SP} $\diamond$ ICMC-USP, São Carlos-SP, Brazil \hfill {\em 2017} \\
Escola Regional de Alto Desempenho do Estado de São Paulo

{\bf SBAC-PAD'19} $\diamond$ Campo Grande-MS, Brazil \hfill {\em 2019} \\
International Symposium on Computer Architecture and High Performance
Computing

{\bf CARLA'22} $\diamond$ Porto Alegre-RS, Brazil \hfill {\em 2022} \\
Latin America High Performance Computing Conference

{\bf CARLA'23} $\diamond$ Cartagena, Colombia \hfill {\em 2023} \\
Latin America High Performance Computing Conference
\end{rSection}

\begin{rSection}{Links}
  \begin{tabular}{ll}
    Github            & {\tt https://github.com/leiteg} \\
    Gitlab            & {\tt https://gitlab.com/leiteg} \\
    Google Scholar    & {\tt https://scholar.google.com/citations?user=F6MZj\_oAAAAJ} \\
    Lattes Curriculum & {\tt https://lattes.cnpq.br/0392351138118593} \\
  \end{tabular}
\end{rSection}

\begin{rSection}{Publications}
\bibliographystyle{abbrv}
\renewcommand{\section}[2]{}
\bibliography{references}
\end{rSection}

\vfill \hfill {\em Last updated: \today.}

\end{document}
